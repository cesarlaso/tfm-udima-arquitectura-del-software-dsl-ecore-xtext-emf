\newacronym{mda}{MDA}{Model Driven Architecture}
\newacronym{mde}{MDE}{Model Driven Enginering}
\newacronym{mdsd}{MDSD}{Model Driven Software Development}
\newacronym{omg}{OMG}{Object Management Group}
\newacronym{cim}{CIM}{Computation-Independent Model}
\newacronym{pim}{PIM}{Platform-Independent Model}
\newacronym{psm}{PSM}{Platform-Specific Model}
\newacronym{qvt}{QVT}{Query/View/Transformation}
\newacronym{uml}{UML}{Unified Modeling Language}
\newacronym{mof}{MOF}{Meta-Object Facility}
\newacronym{xmi}{XMI}{XML Metadata Interchange}
\newacronym{edoc}{EDOC}{Enterprise Distributed Object Computing}
\newacronym{spem}{SPEM}{Software Process Engineering Metamodel}
\newacronym{cwm}{CWM}{Common Warehouse Metamodel}
\newacronym{rpc}{RPC}{Remote Procedure Call}
\newacronym{ocl}{OCL}{Constraint Language Specification}
\newacronym{bpmn}{BPMN}{Business Process Model and Notation}
\newacronym{iot}{IOT}{Internet of things}
\newacronym{sql}{SQL}{Structured Query Language}
\newacronym{gpio}{GPIO}{General purpose input output}
\newacronym{led}{LED}{Light Emitting Diode}
\newacronym{jvm}{JVM}{Java Virtual Machine}
\newacronym{dsl}{DSL}{Domain specific language}
\newacronym{xml}{XML}{Extensible Markup Language}
\newacronym{html}{HTML}{HyperText Markup Language}
\newacronym{json}{JSON}{JavaScript Object Notation}
\newacronym{ide}{IDE}{Integrated Development Environment}
\newacronym{emf}{EMF}{Eclipse Model Framework}
\newacronym{xtext}{Xtext}{Xtext framework for modeling languages}
\newacronym{esa}{ESA}{European Space Agency}
\newacronym{ecore}{ECore}{ECore Eclipse Modeling framework}
\newacronym{tcp}{TCP}{Transmission Control Protocol}
\newacronym{css}{CSS}{Cascading Style Sheets}
\newacronym{rs232}{RS232}{Recommended Standard 232}
\newacronym{i2c}{I2C}{Inter Integrated Circuit}
\newacronym{io}{IO}{Input / Output}
\newacronym{gmp}{GMP}{Graphical modeling proyect}
\newacronym{gmf}{GMF}{Graphical modeling framework}
\newacronym{sirius}{Sirius}{Eclipse Sirius}

\newacronym{api}{API}{Application Programming Interface}

\newglossaryentry{housekeeping}
{
  name=housekeeping,
  description={Comprobación de constantes de salud del sistema}
}

\newglossaryentry{script}
{
        name=script,
        description={Un programa usualmente simple, que por lo regular se almacena en un archivo de texto plano. Los scripts son por lo general siempre interpretados, pero no todo programa interpretado es considerado un script.}
}

\newglossaryentry{zip}
{
        name=zip,
        description={Formato de compresión de archivos sin perdida.}
}


\newglossaryentry{framework}
{
        name=framework,
        description={Conjunto estandarizado de conceptos, prácticas y criterios para enfocar un tipo de problemática particular que sirve como referencia, para enfrentar y resolver nuevos problemas de índole similar}
}

\newglossaryentry{mainloop}
{
        name={Main loop},
        description={Hilo principal de ejecución de eventos. Utilizado normalmente para eventos controlados por el usuario, tales como entradas/salidas, o interfaz gráfico}
}
\newglossaryentry{timestamp}
{
        name={timestamp},
        description={Marca de tiempo, por normal general segundos desde época Unix}
}
\newglossaryentry{payload}
{
        name={payload},
        description={Contenido de un paquete de datos}
}

\newglossaryentry{keepalive}
{
        name={keepalive},
        description={Paquete de datos utilizado para mantener una conexión con estado activa}
}

\newglossaryentry{runtime}
{
        name={runtime},
        description={Código ejecutado en tiempo de ejecución}
}

\newglossaryentry{bytecode}
{
        name={bytecode},
        description={Código intermedio más abstracto que el código máquina. Habitualmente es tratado como un archivo binario que contiene un programa ejecutable similar a un módulo objeto, que es un archivo binario producido por el compilador cuyo contenido es el código objeto o código máquina }
}

\newglossaryentry{wizard}
{
        name={wizard},
        description={Asistente paso a paso utilizado en la creación de elementos}
}

\newglossaryentry{metadatos}
{
        name={metadatos},
        description={Son datos que describen otros datos. En general, un grupo de metadatos se refiere a un grupo de datos que describen el contenido informativo de un objeto al que se denomina recurso}
}

\newglossaryentry{metamodelo}
{
        name={metamodelo},
        description={Modelo que define el lenguaje para expresar un modelo}
}

\newglossaryentry{reflection}
{
        name={reflection},
        description={Propiedad por la cual el propio lenguaje puede inspeccionar y manipular clases e interfaces (así como sus métodos y campos) en tiempo de ejecución, sin conocer a priori (en tiempo de compilación) los tipos y/o nombres de las clases específicas con las que está trabajando}
}
\newglossaryentry{plugin}
{
        name={plugin},
        description={Es una aplicación que se relaciona con otra para agregarle una función nueva y generalmente muy específica. Esta aplicación adicional es ejecutada por la aplicación principal e interactúan por medio de un API.}
}
