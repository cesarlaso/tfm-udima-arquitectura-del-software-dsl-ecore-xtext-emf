\section{Recursos utilizados}

% Esta sección la plantearía como los elementos que has usado para el desarrollo de tu proyecto. Tipos de placas, Ordenador, programas, etc. Esta sección en otras disciplinas tiene más sentido, como medicina, biología,etc, ya que sería material.

Para el desarrollo de este trabajo se han consultado los siguientes libros:

\begin{itemize}

\item \fullcite{Stahl:2006:MSD:1196766} Libro de referencia de la asignatura \textit{Diseño avanzado en arquitecturas software}.
\item \fullcite{dsl_xtext_xtend}.
\item \fullcite{models_to_code}.
\item \fullcite{mda_distilled}.

\end{itemize}

Como material hardware se ha contado con el siguiente material para realizar pruebas:

\begin{itemize}
\item Raspberry PI \cite{raspberrypi} modelo B. Para el desarrollo de la funcionalidad IOT en java.
\item ESPDuino (Modulo ESP8266 compatible Arduino). Para la comprobación de la funcionalidad en Arduino y freeRTOS.
\end{itemize}

Herramientas informáticas entornos IDE:
\begin{itemize}
\item Eclipse \cite{Eclipse} EMF \cite{EclipseEMF} para \gls{ecore}, Xtext, Xtend.
\item Netbeans para Java. Para la edición de código Java.
\item \gls{ide} Arduino \cite{arduino}. Para la realización de pruebas en código C y extraer la información necesaria para posteriormente generar modelos compatibles con los modelos Java.
\end{itemize}

Editores de texto y compilación de esta memoria en formato PDF:
\begin{itemize}
\item Latex \cite{Latex}, entorno completo para Linux y Mac.
\item ShareLatex.com \cite{sharelatex}. Edición y compilación online de archivos Latex.
\item Adobe AcrobatReader. Para la visualización y comprobación de la memoria.
\end{itemize}

Máquinas virtuales:
\begin{itemize}
\item Java SE
\end{itemize}


El resto de recursos que puntualmente han sido utilizados serán anotados en sus correspondientes secciones y posteriormente al final del trabajo en la sección bibliografía.
